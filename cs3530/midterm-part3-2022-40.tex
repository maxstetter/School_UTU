%
% Exam for CS3530 Computational Theory:
% Fall 2022
\documentclass{article}

\usepackage[margin=1in]{geometry}
\usepackage{amsfonts}
\usepackage{amsmath}
\usepackage[english]{babel}
\usepackage[utf8]{inputenc}
\usepackage{ae,aecompl}
\usepackage{emp,ifpdf}
\usepackage{graphicx}
\usepackage{enumerate}

% skip for paragraphs, don't indent
\parskip 6pt plus 1pt
\parindent=0pt
\raggedbottom

\begin{document}
\begin{empfile}

\begin{center}
\textbf{\Large Exam for CS 3530:~Computational Theory} \\[2mm]
December 1-7, 2022 \\[2mm]

Your Name Here

\textit{You have until 11:59 pm on December 7 to submit your solution. Points total 30.
Your notes, books, and homework are allowed. You may use any theorem or lemma
already proved in a lecture, in the textbook, or on a problem set.  Submitting
work that is not yours is strictly forbidden.}
\end{center}

\raggedright

\subsection*{Instructions}

In each problem below, complete the required task.  You must follow the
instructions given. If the instructions are unclear, please ask for clarification.

\subsection*{Problem 1 (10 points)}

Let $\textsc{Triple-Sat} = \{ \phi | \phi $ has at least three satisfying assignments $\}$. Show that
\textsc{Triple-Sat} is NP-complete.

\emph{Hint:} Consider a reduction from \textsc{Sat} or \textsc{3Sat}.

\emph{Note:} In order to receive credit for this problem, you must complete the full NP-completeness proof process outlined in assignments.

\subsection*{Solution}


\subsection*{Problem 2 (10 points)}

Let $\textsc{Half-Dominating-Set} = \{ \langle G \rangle | G $ has a dominating set of size $m/2$, where $m$ is the number of nodes in $G$ $\}$.
Show that \textsc{Half-Dominating-Set} is NP-complete.

\emph{Hint:} Consider a reduction from \textsc{Dominating-Set}.

\emph{Note:} In order to receive credit for this problem, you must complete the full NP-completeness proof process outlined in assignments.

\subsection*{Solution}


\subsection*{Problem 3 (10 points)}

Let $\textsc{Subset-Sum} = \{ \langle S,t \rangle | S = \{ x_1,...,x_k\}$, $t$ and each $x_i$ are integers,
and for some $\{y_1,...,y_l\} \subseteq \{ x_1,...,x_k\}$, $\Sigma y_i = t$  $\}$.
Show that \textsc{Subset-Sum} is NP-complete.

\emph{Hint:} Consider looking at Theorems 7.25 and 7.56 in the textbook.

\emph{Note:} In order to receive credit for this problem, you must complete the full NP-completeness proof process outlined in assignments.


\subsection*{Solution}





\end{empfile}
\immediate\write18{mpost -tex=latex \jobname}
\end{document}
